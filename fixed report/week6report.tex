\documentclass[11pt]{article}

\usepackage{times}
\usepackage[english]{babel}

% -----------------------------------------------
% especially use this for you code
% -----------------------------------------------

\usepackage{courier}
\usepackage{listings}
\usepackage{color}
\usepackage{tabularx}
\usepackage{graphicx}

\definecolor{Gray}{gray}{0.95}

\definecolor{mygreen}{rgb}{0,0.6,0}
\definecolor{mygray}{rgb}{0.5,0.5,0.5}
\definecolor{mymauve}{rgb}{0.58,0,0.82}

\lstset{language=C++,
	basicstyle = \normalsize\ttfamily,   % the size and fonts that are used
	tabsize = 2,                    % sets default tabsize
	breaklines = true,              % sets automatic line breaking
	keywordstyle=\color{blue}\ttfamily,
	stringstyle=\color{red}\ttfamily,
	commentstyle=\color{mygreen}\ttfamily,
	numbers=left,
	keepspaces=true,
	showspaces=false,
	showstringspaces=false,
}

\begin{document}

\title{Programming in C/C++ \\
       Exercises set six: parsers II
}
\date{\today}
\author{Christiaan Steenkist \\
Jaime Betancor Valado \\
Remco Bos \\
}

\maketitle
\section*{Exercise 37, substantial grammar extension}
All these operators.

\subsection*{Code listings}
\subsubsection*{Scanner}
\lstinputlisting[caption = lexer]{src/a37/scanner/lexer}

\subsubsection*{Parser}
%\lstinputlisting[caption = grammar]{src/a37/parser/grammar}
%\lstinputlisting[caption = parser.h]{src/a37/parser/parser.h}
\lstinputlisting[caption = toint.cc]{src/a37/parser/toint.cc}

\section*{Exercise 39, functions}
This was actually made before 36-38.

\subsection*{Code listings}
\lstinputlisting[caption = grammar]{src/a39/parser/grammer.gr}
\subsubsection*{parser.h snippet}
\begin{lstlisting}
	// arithmetic functions:
	
		void display(double &value);
		void done();
		void prompt();
		RuleValue &exp(RuleValue &value);
		RuleValue &ln(RuleValue &value);
		RuleValue &sin(RuleValue &value);
		RuleValue &asin(RuleValue &value);
		RuleValue &sqrt(RuleValue &value);
		RuleValue &abs(RuleValue &value);
		
		RuleValue &deg(RuleValue &value);		
		RuleValue &grad(RuleValue &value);
		RuleValue &rad(RuleValue &deg);
		RuleValue &rad(RuleValue &grad);
		
		double const pi = 3.14159;
		double const e = 2.71828;
\end{lstlisting}

\subsubsection*{Implementations}
\lstinputlisting[caption = abs.cc]{src/a39/parser/abs.cc}
\lstinputlisting[caption = asin.cc]{src/a39/parser/asin.cc}
\lstinputlisting[caption = deg.cc]{src/a39/parser/deg.cc}
\lstinputlisting[caption = done.cc]{src/a39/parser/done.cc}
\lstinputlisting[caption = exp.cc]{src/a39/parser/exp.cc}
\lstinputlisting[caption = grad.cc]{src/a39/parser/grad.cc}
\lstinputlisting[caption = ln.cc]{src/a39/parser/ln.cc}
\lstinputlisting[caption = raddeg.cc]{src/a39/parser/raddeg.cc}
\lstinputlisting[caption = radgrad.cc]{src/a39/parser/radgrad.cc}
\lstinputlisting[caption = sin.cc]{src/a39/parser/sin.cc}
\lstinputlisting[caption = sqrt.cc]{src/a39/parser/sqrt.cc}

\section*{Exercise 40, polymorphic value type class}
We attempted to make a polymorphic value type class.

\subsection*{Code listings}
\lstinputlisting[caption = grammar.gr]{src/a40/grammar.gr}
\lstinputlisting[caption = Parser.ih]{src/a40/Parser.ih}
\lstinputlisting[caption = Parser.h]{src/a40/Parser.h}
\lstinputlisting[caption = getdouble.cc]{src/a40/getdouble.cc}
\lstinputlisting[caption = getint.cc]{src/a40/getint.cc}
\lstinputlisting[caption = getstring.cc]{src/a40/getstring.cc}
\lstinputlisting[caption = quit.cc]{src/a40/quit.cc}
\lstinputlisting[caption = showdouble.cc]{src/a40/showdouble.cc}
\lstinputlisting[caption = showint.cc]{src/a40/showint.cc}
\lstinputlisting[caption = showstring.cc]{src/a40/showstring.cc}

\subsubsection*{Polymorphic type}
\lstinputlisting[caption = polytype.ih]{src/a40/polytype.ih}
\lstinputlisting[caption = polytype.h]{src/a40/polytype.h}
\lstinputlisting[caption = inttype\_constr.cc]{src/a40/inttype_constr.cc}
\lstinputlisting[caption = inttype\_print.cc]{src/a40/inttype_print.cc}

\end{document}